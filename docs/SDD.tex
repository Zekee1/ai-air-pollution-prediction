\documentclass{article}

\title{Software Design Document (SDD) \\ AI Air Pollution Prediction & Visualization System}
\author{Ezequiel Flores, Ashley Jones, Joseph Vazquez, Kevin Antezana}
\date{Snapshot 1}

\begin{document}

\maketitle

\tableofcontents

\section*{Version Table}
\begin{tabular}{|c|c|c|}
\hline
Version & Description & Date \\
\hline
1.0 & Initial SDD for Snapshot 1 & \today \\
\hline
\end{tabular}

\section{Introduction}
\subsection{Purpose}
This document outlines the design of an AI-powered system that predicts air pollution levels and visualizes environmental data.

\subsection{Intended Audience}
Professor, development team, data scientists, testers.

\subsection{Overview}
This SDD describes the architecture, interface, and main components of the system.

\section{System Architecture}
\subsection{High-Level Workflow}
\begin{itemize}
    \item User accesses dashboard
    \item Dashboard requests data from backend
    \item Data pipeline loads pollution dataset
    \item AI model generates predictions
    \item Visualizations are displayed to user
\end{itemize}

\subsection{Components}
\begin{itemize}
    \item Data ingestion module
    \item Data cleaning module
    \item Machine learning model
    \item Visualization/dashboard interface
\end{itemize}

\subsection{Database Design}
For Snapshot 1, the system will use CSV files as the primary data source for storing air pollution measurements. 
This includes attributes such as PM2.5, PM10, CO, and NO2 levels. 
Future versions of the system may include a database such as SQLite or PostgreSQL to store user-uploaded data, prediction results, and logs.

\section{User Interface}
\begin{itemize}
    \item Dashboard home page
    \item Real-time pollution chart
    \item Prediction visualization page
    \item Data upload/viewing page
\end{itemize}

\subsection{How to Use the System}
Users will interact with the system through a web-based dashboard. 
They can upload air quality datasets, view automatically generated visualizations, 
and request predictions from the AI model. 
The dashboard will guide the user through selecting parameters and viewing results.

\section{Glossary}
\begin{itemize}
    \item AI - Artificial Intelligence
    \item ML - Machine Learning
    \item PM2.5 - Fine particulate matter
\end{itemize}

\section{References}
\begin{itemize}
    \item Kaggle Air Quality Dataset
    \item Scikit-Learn Documentation
\end{itemize}

\end{document}
