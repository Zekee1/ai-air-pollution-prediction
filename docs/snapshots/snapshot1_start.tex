\documentclass{article}

\title{Snapshot 1 Objective \\ AI Air Pollution Prediction \& Visualization System}
\author{Ezequiel Flores, Ashley Jones, Joseph Vazquez, Kevin Antezana}
\date{Snapshot 1}

\begin{document}

\maketitle

\section*{Purpose of Snapshot 1}
The purpose of Snapshot 1 is to set up the foundation of the AI Air Pollution Prediction and Visualization System by defining the core idea, goals, 
structure, and initial planning for the project. This snapshot focuses on laying out the base framework, selecting dependencies,
planning tasks, and organizing the documentation required for the rest of the development process.

\section*{Goals for the Start of the Project}
\begin{itemize}
    \item Define the overall concept and problem the software aims to solve.
    \item Identify the primary objectives of the project, including AI-based pollution prediction and data visualization.
    \item Determine the core features and essential components for the system.
    \item Establish initial dependencies and tools such as Python, Pandas, NumPy, Scikit-Learn, and Streamlit/Flask.
    \item Create the documentation for SDD, SRS, and User Manual (README) as required.
    \item Set up the initial workflow and system architecture for how components interact.
    \item Organize group roles and task other members for upcoming snapshots.
    \item Prepare the GitHub repository structure and include all initial planning material.
\end{itemize}

\section*{Expected Outcomes of Snapshot 1}
By the end of Snapshot 1, the team will have:
\begin{itemize}
    \item A complete GitHub repository containing all initial documents (SDD, SRS, README).
    \item A clear high-level workflow diagram explaining how users interact with the system.
    \item A defined list of tools, frameworks, and base dependencies.
    \item A structured snapshot plan for future checkpoints.
    \item A clear foundation to begin development in Snapshot 2.
\end{itemize}

\end{document}
